\documentclass{article}
\usepackage{amsmath, amssymb, amsthm}
\usepackage{hyperref}

\title{Introduction to Calculus}
\author{Your Name}
\date{\today}

\begin{document}
\maketitle

\section{Limits and Continuity}
Calculus begins with the concept of limits and continuity.

\subsection{Limit}
The limit of a function \(f(x)\) at a point \(x = a\) is defined as:
\[
\lim_{{x \to a}} f(x) = L
\]
if for every \(\varepsilon > 0\), there exists a \(\delta > 0\) such that if \(0 < |x - a| < \delta\), then \(|f(x) - L| < \varepsilon\).

\subsection{Continuity}
A function \(f(x)\) is continuous at a point \(x = a\) if the following three conditions hold:
\begin{enumerate}
    \item \(f(a)\) is defined.
    \item \(\lim_{{x \to a}} f(x)\) exists.
    \item \(\lim_{{x \to a}} f(x) = f(a)\).
\end{enumerate}

\section{Derivatives}
The derivative of a function measures its rate of change.

\subsection{Derivative Definition}
The derivative of a function \(f(x)\) at a point \(x = a\) is defined as:
\[
f'(a) = \lim_{{h \to 0}} \frac{f(a + h) - f(a)}{h}
\]

\subsection{Differentiability}
A function \(f(x)\) is differentiable at a point \(x = a\) if the derivative \(f'(a)\) exists.

\section{Integrals}
Integrals are used to calculate the accumulated area under a curve.

\subsection{Definite Integral}
The definite integral of a function \(f(x)\) over the interval \([a, b]\) is denoted as:
\[
\int_{{a}}^{{b}} f(x) \, dx
\]

\subsection{Indefinite Integral}
The indefinite integral of a function \(f(x)\) is denoted as:
\[
\int f(x) \, dx
\]

\section{Conclusion}
Calculus is a fundamental branch of mathematics that deals with limits, continuity, derivatives, and integrals. These concepts form the basis for understanding changes and accumulation in various mathematical and scientific fields.

\end{document}

