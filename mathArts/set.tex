\documentclass{article}
\usepackage{amsmath, amssymb}
\usepackage{enumitem}
\usepackage{hyperref}

\title{Introduction to Set Theory}
\author{Your Name}
\date{\today}

\begin{document}
\maketitle

\section{Introduction}

Set theory is a fundamental branch of mathematics that deals with the study of sets, which are collections of distinct objects. The ideas of set theory have been around for centuries, but it was in the late 19th and early 20th centuries that mathematicians formalized the subject, laying down its axioms and principles. Set theory serves as the foundation for much of modern mathematics, providing the language and framework for discussions about mathematical objects and structures.

\section{Basic Definitions}

\begin{enumerate}[label=(\alph*)]
    \item \textbf{Set}: A set is a well-defined collection of distinct objects. We denote a set by listing its elements inside curly braces, e.g., $A = \{1, 2, 3\}$.
    
    \item \textbf{Element}: An element is an object that belongs to a set. If $x$ is an element of set $A$, we write $x \in A$.
    
    \item \textbf{Subset}: A set $B$ is a subset of set $A$ (denoted as $B \subseteq A$) if every element of $B$ is also an element of $A$.
    
    \item \textbf{Union}: The union of two sets $A$ and $B$, denoted as $A \cup B$, is the set containing all elements that belong to either $A$, or $B$, or both.
    
    \item \textbf{Intersection}: The intersection of two sets $A$ and $B$, denoted as $A \cap B$, is the set containing all elements that belong to both $A$ and $B$.
    
    \item \textbf{Complement}: The complement of a set $A$ with respect to a universal set $U$, denoted as $A'$ or $\bar{A}$, is the set containing all elements in $U$ that are not in $A$.
\end{enumerate}

\section{Quiz}

Test your understanding with the following definitions quiz:

\begin{enumerate}
    \item What is a set?
    \item If $x$ is an element of set $A$, how is this denoted?
    \item Define the term "subset."
    \item What is the union of two sets $A$ and $B$?
    \item Define the term "intersection."
    \item What is the complement of a set $A$?
\end{enumerate}

\section{Conclusion}

Set theory is a foundational topic in mathematics that provides the tools for understanding the relationships between different collections of objects. By mastering the basic definitions and concepts of set theory, you'll be better equipped to delve into more advanced mathematical topics.

\end{document}
